\begin{table*}
\setlength{\tabcolsep}{4pt} 
\begin{tabular}{p{1.6cm} l p{8.2cm} p{.6cm}  p{.8cm} p{.8cm}}
\hline
Team     & Model Abbr& Model Description & Ext. Data & Mech. Model & MM Ens. \\ 
\hline

FSNetwork & EW       & Equal Weights (number of estimated weights = 0) & ~  & & x\\
 & CW       & Constant Weights (20)& & & x  \\
 & TTW      & Target-Type Weights (40) & & & x \\ 
 & TW       & Target Weights (140) & & & x \\ 
 & TRW      & Target-Region Weights (1,540) & & & x \\ 
\hline
CU       & EAKFC\_SEIRS$^\dagger$       & Ensemble Adjustment Kalman Filter SEIRS\cite{Pei2017}  & x & x & \\ 

~        & EAKFC\_SIRS$^\dagger$        & Ensemble Adjustment Kalman Filter SIRS\cite{Pei2017}  & x & x & \\
~        & EKF\_SEIRS$^\dagger$         & Ensemble Kalman Filter SEIRS\cite{Yang2014}    & x             & x &                    \\
~        & EKF\_SIRS$^\dagger$          & Ensemble Kalman Filter SIRS\cite{Yang2014}     & x             & x   &                 \\
~        & RHF\_SEIRS$^\dagger$         & Rank Histogram Filter SEIRS\cite{Yang2014}    & x             & x     &               \\
~        & RHF\_SIRS$^\dagger$          & Rank Histogram Filter SIRS\cite{Yang2014}     & x             & x       &             \\
~        & BMA                & Bayesian Model Averaging\cite{Yamana2017}      & ~             & ~          &          \\
\hline
Delphi   & BasisRegression*    & Basis Regression ({\tt epiforecast} defaults)\cite{Brooks2015a} & ~             & ~     &               \\ 
~        & DeltaDensity1*      & Delta Density ({\tt epiforecast} defaults)\cite{Brooks2018} & ~             & ~       &             \\ 
~        & EmpiricalBayes1*    & Empirical Bayes (conditioning on past 4 weeks)\cite{Brooks2015,Brooks2015a} & ~             & ~  &                  \\ 
~        & EmpiricalBayes2*    & Empirical Bayes ({\tt epiforecast} defaults)\cite{Brooks2015,Brooks2015a} & ~             & ~            &        \\ 
~        & EmpiricalFuture*    & Empirical Futures ({\tt epiforecast} defaults)\cite{Brooks2015a} & ~             & ~         &           \\ 
~        & EmpiricalTraj*      & Empirical Trajectories ({\tt epiforecast} defaults)\cite{Brooks2015a} & ~             & ~         &           \\ 
~        & DeltaDensity2*      & Markovian Delta Density ({\tt epiforecast} defaults)\cite{Brooks2018} & ~             & ~          &          \\ 
~        & Uniform*            & Uniform Distribution& & ~   &                 \\ 
~        & Stat               & Ensemble (combination of 8 Delphi models)\cite{Brooks2018} & ~             & ~  & x                 \\
\hline
LANL     & DBM                & Dynamic Bayesian SIR Model with discrepancy\cite{osthus2018dynamic} & ~             & x      &              \\ 
\hline
ReichLab & KCDE               & Kernel Conditional Density Estimation\cite{Ray2017}  & ~             & ~            &        \\ 
~        & KDE                & Kernel Density Estimation and penalized splines\cite{Ray2018}  & ~             & ~     &               \\ 
~        & SARIMA1            & SARIMA model without seasonal differencing\cite{Ray2018} & ~             & ~      &              \\ 
~        & SARIMA2            & SARIMA model with seasonal differencing\cite{Ray2018} & ~             & ~           &         \\ 
\hline
FluSight & unweighted\_avg       & Average of all models submitted to the CDC\cite{McGowan2018}  &   &  &  x\\
\end{tabular} 
\caption{List of models, with key characteristics. New ensemble models introduced by this paper are indicated with the prefix FSNetwork. Component models contributed by individual teams are grouped by team with team-specific prefixes as follows: CU = Columbia University, Delphi = Carnegie Mellon, LANL = Los Alamos National Laboratory, ReichLab = University of Massachusetts Amherst, FluSight = CDC challenge organizers. The FluSight model was not included in the collaborative multi-model ensemble, but is used as a reference multi-model ensemble in the analysis. The `Ext data' column notes models that use data external to the ILINet data from CDC. The `Mech. model' column notes models that rely to some extent on a mechanistic or compartmental model of infectious disease transmission.\cite{keeling2011modeling} The `MM Ens.' column indicates models that are multi-model ensembles. Note that some of the components were not designed as standalone models (marked with *) and others used single-model ensemble methodologies (marked with $\dagger$) (see Methods for more details). S(E)IRS abbreviations stand for Susceptible (Exposed) Infectious Recovered Susceptible models of disease transmission and SARIMA stands for Seasonal Auto-Regressive Integrated Moving Average Model (see references for details).}
\label{tab:model-list}
\end{table*}